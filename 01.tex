\documentclass[a4paper,12pt,numbers=noenddot]{scrreport}

\usepackage[german]{datetime2}
\usepackage[onehalfspacing]{setspace}
\usepackage[utf8]{inputenc}
\usepackage{amsmath, amsfonts, amssymb, amsthm}
\usepackage{mathrsfs}

\usepackage{scrlayer-scrpage}
\renewcommand*{\chapterpagestyle}{scrheadings}
\clearpairofpagestyles

\ohead{\normalfont \today}
\chead{\normalfont Algebraic Automata Theory}
\ihead{\normalfont \rightmark}

\ofoot{\normalfont Seite~\pagemark}
\cfoot{\normalfont CAU Kiel - Technische Fakultät}
\ifoot{\normalfont Eric Hotho, Klaas Pelzer}

\KOMAoptions{headsepline=true,footsepline=true}
\renewcommand*{\chapterheadstartvskip}{\vspace*{-.4cm}}
\renewcommand*{\chapterheadendvskip}{\vspace{.5cm}}

\setlength{\parindent}{0pt}

\setkomafont{chapter}{\LARGE}
\setkomafont{section}{\Large}
\setkomafont{subsection}{\large}
\setkomafont{subsubsection}{\normalsize}
\setkomafont{paragraph}{\normalsize}
\setkomafont{subparagraph}{\small}
\usepackage[htt]{hyphenat}
\usepackage[german]{babel}
\setcounter{chapter}{0}

\begin{document}
\automark{section}
\automark{chapter}
\chapter{}
\section{}
We show that $R^{-1} \subseteq Y \times X$ is a partial function if $R \subseteq X \times Y$ is injective by proofing $R^{-1}$ satisfies the right uniquess condition
\begin{align*}
    (y,x), (y,x') \in R^{-1} \Rightarrow x = x'.
\end{align*}
Since R is injective the following holds true:
\begin{align*}
    (x,y), (x',y') \in R | x \neq x' \Rightarrow y \neq y'\label{eq0}\tag{1}
\end{align*}
Thus 
\begin{align*}
    (y,x), (y,x') \in R^{-1} | (x,y), (x', y) \in R  \Rightarrow x = x'
\end{align*}
because $x \neq x'$ contradicts \eqref{eq0}.
\qed

\section{}
$\emptyset \subseteq \emptyset \times Y$:
First we show that $\emptyset \subseteq \emptyset \times Y$ is a partial function.
By definition $$(x,y), (x',y') \in R \Rightarrow y = y'$$ the empty relation is right-unique because we have no elements in the relation.
Further we show totalness with $dom(R) = \emptyset$.
By Definition
$$dom(R) = \{a \in A | \exists a' \in A: (a,a') \in R\}$$
the domain of the empty relation is also empty due missing elements and therfore $dom(R) = \emptyset = \emptyset$.

\vspace{1cm}
$\emptyset \subseteq X \times \emptyset$: This relation is also a partial function because the relation is also empty.
But it is not total because the preimage is $X$ and the domain is $\emptyset$.
Therefore
$$dom(R) = \emptyset \neq X$$
\qed

\section{}
Let $(S, \cdot)$, $(T, *)$ be two semigroups and $f: S \rightarrow T$ be a partial semigroup homomorphismus, then we have $f(S) \leq_{sg} T$ with
\begin{align}
    f(S) * f(s') = f(s, s') \label{eq1}\tag{1}
\end{align}
For $f(S)$ to be subsemigroup of $T$ iff $f(S)$ is closed under $*$.
For it to be closed under $*$ means that 
$$\forall x,y \in f(S) : x * y \in f(S)$$
Since $f$ is a semigroup homomorphismus \eqref{eq1} holds true, thus
\begin{align}
    x * y = f(f^{-1}(x) \cdot f^{-1}(y)) \label{eq2}\tag{2}
\end{align}
and since $(S, \cdot)$ is a semigroup, it is closed under $\cdot$ therefore
$$f^{-1}(x) \cdot f^{-1}(y) \in S$$
and with this \eqref{eq2} holds true and therefore the initial implication of Lemma 2.49.
\qed

\section{}

\end{document}
