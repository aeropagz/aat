\documentclass[a4paper,12pt,numbers=noenddot]{scrreport}

\usepackage[german]{datetime2}
\usepackage[onehalfspacing]{setspace}
\usepackage[utf8]{inputenc}
\usepackage{amsmath, amsfonts, amssymb, amsthm}
\usepackage{mathrsfs}

\usepackage{scrlayer-scrpage}
\renewcommand*{\chapterpagestyle}{scrheadings}
\clearpairofpagestyles

\ohead{\normalfont \today}
\chead{\normalfont Algebraic Automata Theory}
\ihead{\normalfont \rightmark}

\ofoot{\normalfont Seite~\pagemark}
\cfoot{\normalfont CAU Kiel - Technische Fakultät}
\ifoot{\normalfont Eric Hotho, Klaas Pelzer}

\KOMAoptions{headsepline=true,footsepline=true}
\renewcommand*{\chapterheadstartvskip}{\vspace*{-.4cm}}
\renewcommand*{\chapterheadendvskip}{\vspace{.5cm}}

\setlength{\parindent}{0pt}

\setkomafont{chapter}{\LARGE}
\setkomafont{section}{\Large}
\setkomafont{subsection}{\large}
\setkomafont{subsubsection}{\normalsize}
\setkomafont{paragraph}{\normalsize}
\setkomafont{subparagraph}{\small}
\usepackage[htt]{hyphenat}
\usepackage[german]{babel}
\setcounter{chapter}{1}

\begin{document}
\automark{section}
\automark{chapter}
\chapter{}
\section{}
Since the group is finite, we can generate the powers of an element $s \in S$ infinetly.
However there a only a finite amount of elements in $S$. 
Thus, there exists an integer $x$ such that $s^x=s$. 
If $x=2$ we are done, else we multiply both sides with $s^{x-2}$.

\begin{align*}
    s^2 \cdot s^{x-2} &= s \cdot s^{x-2} \\
    (s^{x} \cdot s^{-1}) \cdot (s^{x} \cdot s^{-1}) &= s^{x-1} \\
    s^{x-1} \cdot s^{x-1} &= s^{x-1} \\
    (s^{x-1})^2 &= s^{x-1} \\
\end{align*}
Then we substitute $s^{(x-1)}$ with $y$ and the result is $y^2 = y'$.
\qed

\section{}
First we proof $g,g' \in G, g \sim_H g'$ if $g = hg'$ is an equivalence relation by showing it is reflexive, simmetric and transitive.

Reflexivity:
For all $g \in G$ we have to prove that $g = hg'$, since $H$ is a Subgroup of $G$, it contains the neutral Element of $G$ therefore,
we can always choose the neutral element for $h$ with $g = hg'$. Thus
$g = g'$ and $g \sim_H g$.

Symmetry:
Assume $g \sim_H g'$ with $g = hg'$. Then we can add in inverse element $h^{-1}$, which is also in $H$ because it is a group, resulting in $h^{-1}g'= g$.
Hence, $g' \sim_H g$ is also true.

Transitivity:
Assume $g \sim_H g'$ and $g' \sim_H g''$ with $g'' \in G$ and $h_1, h_2 \in H: g = h_1g'$, $g' = h_2g''$.
We can substitute $g'$ resulting in $g = h_1h_2g''$. Since $H$ is closed $h_1h_2$ is also in $H$. 
Hence, $g \sim_H g''$ is also true.

Since we proofed that $\sim_H$ is a equivalence relation, $G/H$ is partition induced by $\sim_H$.
\qed

\section{}
We proof if $G$ is a permutation group on $Q$ then $G$ acts on $Q$.
Therefore we show that $G$ fulfills the two conditions
\begin{align*}
    & \forall q \in Q, g_1,g_2 \in G: q(g_1g_2) = (qg_1)g_2 \label{eq0}\tag{1} \\
    & g_1,g_2 \in G: \text{if } qg_1 = qg_2 \text{ for all } q \in Q \text{ then } g_1 = g_2 \label{eq1}\tag{2} \\
\end{align*}
Hence, G is a group by Lemma 2.89 it is associative and satisfies \eqref{eq0}.
Also permutations are bijective which satisfies \eqref{eq1}.
\qed

\section{}

First we prove that $f$ is a semigroup homomorphism. For this the following hast to hold true:
\begin{equation}
    f(s) * f(s') = f(s \cdot s')
\end{equation}

given $w, w' \in \Sigma ^* with |w| = n, |w'| = m : n,m \in \mathbb{N\textsubscript{0}}$
\begin{equation}
    f(w)+f(w') = n+m = f(ww')
\end{equation}
To now prove that $f$ is a monoid homomorphism we show that f(e) = e' with $e \in \Sigma ^* \land e' \in \mathbb{N\textsubscript{0}}$ with $e$ and $e'$ being the respective neutral elements. THe neutral element of $\Sigma ^*$ is $\epsilon$ the empty word which has a length of $0$, which is the neutral element of the $\mathbb{N\textsubscript{0}}$, therefore, the mapping of $f$ is correct.\\

For a given word $w \in \Sigma ^*$ its equivalence class is $[|w|]$ with all other words with the same length.
Since the words of $\Sigma ^*$ can be arbitrary long, the order of the quotion set is infinte.
\qed
\end{document}
