\documentclass[a4paper,12pt,numbers=noenddot]{scrreport}

\usepackage[german]{datetime2}
\usepackage[onehalfspacing]{setspace}
\usepackage[utf8]{inputenc}
\usepackage{amsmath, amsfonts, amssymb, amsthm}
\usepackage{mathrsfs}

\usepackage{scrlayer-scrpage}
\renewcommand*{\chapterpagestyle}{scrheadings}
\clearpairofpagestyles

\ohead{\normalfont \today}
\chead{\normalfont Algebraic Automata Theory}
\ihead{\normalfont \rightmark}

\ofoot{\normalfont Seite~\pagemark}
\cfoot{\normalfont CAU Kiel - Technische Fakultät}
\ifoot{\normalfont Eric Hotho, Klaas Pelzer}

\KOMAoptions{headsepline=true,footsepline=true}
\renewcommand*{\chapterheadstartvskip}{\vspace*{-.4cm}}
\renewcommand*{\chapterheadendvskip}{\vspace{.5cm}}

\setlength{\parindent}{0pt}

\setkomafont{chapter}{\LARGE}
\setkomafont{section}{\Large}
\setkomafont{subsection}{\large}
\setkomafont{subsubsection}{\normalsize}
\setkomafont{paragraph}{\normalsize}
\setkomafont{subparagraph}{\small}
\usepackage[htt]{hyphenat}
\usepackage[german]{babel}
\setcounter{chapter}{2}

\def\lsk{\left<}
\def\rsk{\right>}

\begin{document}
\automark{section}
\automark{chapter}
\chapter{}
\section{}
\section{}
\section{}
\begin{center}
\begin{tabular}{c|cc|l}
x & $q_0$ & $q_1$   \\ \hline
a & $q_0$ & $q_1$   \\
b & $q_1$ & $\bot$  \\
c & $\bot$ & $q_0$  \\ \hline
aa & $q_0$ & $q_1$ & same as a \\ 
ab & $q_1$ & $\bot$ & same as b \\ 
ac & $\bot$ & $q_0$ & same as c \\ 
ba & $q_1$ & $\bot$ & same as b \\ 
bb & $\bot$ & $\bot$ \\ 
bc & $q_0$ & $\bot$ \\ 
ca & $\bot$ & $q_0$ & same as c \\ 
cb & $\bot$ & $q_1$ \\ 
cc & $\bot$ & $\bot$ & same as bb \\ \hline
$\vdots$ & $\vdots$ & $\vdots$ & \\
\end{tabular}
\end{center}
From the table we compute $S(M) = \{[a], [b], [c], [b^2], [bc], [cb]\}$.
Suppose $S(M)$ is a group with $u,v \in \Sigma^*, \delta_u, \delta_v \in \lsk \mathcal{F}(M) \rsk$ then there exists for an arbitrary but fixed $\delta_u$ another $\delta_v$ with $\delta_u \delta_v = id_Q$.
Notice we can choose $q_1\delta_b = \bot$ and we cannot find any $\delta_v$ with $\bot \delta_v = q_1$.
Therefore $S(M)$ is not a group.

Now assume $S(M)$ is a monoid.
Hence, there exists $u \in \Sigma^*, \delta_u \in \lsk \mathcal{F}(M) \rsk$ with $\delta_u = id_Q$.
Notice $\delta_a$ satisfies this condition and acts a neutral element.
Thus, $S(M)$ is a monoid.
\qed

\section{}
\end{document}
