\documentclass[a4paper,12pt,numbers=noenddot]{scrreport}

\usepackage[german]{datetime2}
\usepackage[onehalfspacing]{setspace}
\usepackage[utf8]{inputenc}
\usepackage{amsmath, amsfonts, amssymb, amsthm}
\usepackage{mathrsfs}

\usepackage{scrlayer-scrpage}
\renewcommand*{\chapterpagestyle}{scrheadings}
\clearpairofpagestyles

\ohead{\normalfont \today}
\chead{\normalfont Algebraic Automata Theory}
\ihead{\normalfont \rightmark}

\ofoot{\normalfont Seite~\pagemark}
\cfoot{\normalfont CAU Kiel - Technische Fakultät}
\ifoot{\normalfont Eric Hotho, Klaas Pelzer}

\KOMAoptions{headsepline=true,footsepline=true}
\renewcommand*{\chapterheadstartvskip}{\vspace*{-.4cm}}
\renewcommand*{\chapterheadendvskip}{\vspace{.5cm}}

\setlength{\parindent}{0pt}

\setkomafont{chapter}{\LARGE}
\setkomafont{section}{\Large}
\setkomafont{subsection}{\large}
\setkomafont{subsubsection}{\normalsize}
\setkomafont{paragraph}{\normalsize}
\setkomafont{subparagraph}{\small}
\usepackage[htt]{hyphenat}
\usepackage[german]{babel}
<<<<<<< HEAD
\setcounter{chapter}{0}
=======
\setcounter{chapter}{3}

\def\lsk{\left<}
\def\rsk{\right>}
>>>>>>> 55c3a640b193821d34f66700df679210644d300c

\begin{document}
\automark{section}
\automark{chapter}
<<<<<<< HEAD

\setcounter{chapter}{3}
\chapter{}
\section{}

We can find a state machine homomorphism (SMH) for $M$ and $M'$ which is monomorphic but not epimorphic or isomorphic. To prove this conjecture we will first show that there can not exist a epimorphic SMH. For a SMH $f$ to be epimorphic it has to be surjective.
=======
\chapter{}
\section{}
We can find a state machine homomorphism (SMH) for $M$ and $M'$ which is monomorphic but not epimorphic or isomorphic.
To prove this conjecture we will first show that there can not exist a epimorphic SMH. For a SMH $f$ to be epimorphic it has to be surjective.
>>>>>>> 55c3a640b193821d34f66700df679210644d300c
This means that given $f=(\alpha, \beta)$:
\begin{equation}
    \alpha : Q \rightarrow Q' \land \beta : \Sigma \rightarrow \Sigma'
\end{equation}
the range of $\alpha$ and $\beta$ have to be equal to $Q'$ and $\Sigma'$ respectively.

<<<<<<< HEAD
Lets assume that $\alpha$ maps to $Q'$ and $\beta$ to $\Sigma'$. This means that we have to map the states of $Q$ to the states of $Q'$. The state $r\textsubscript{1}$ is isolated whilst the states $r\textsubscript{2}$ and $r\textsubscript{3}$ form a cyclic state machine with stern with stern length and cycle length of $1$. Therefore, we would need to find a similar structure in $Q$ to map the states to $Q'$. If we look for these structures in $M$ we can find an isolated state $q\textsubscript{4}$ when we disregard $c$. However, we can only find a cyclic state machine with stern if we either disregard $a,c$ or $b,c$ since the combination of $b$ and $c$ allows us to always move between the states of $q\textsubscript{i}$ $| i \in {3}$. Thus, we can not find a mapping of $\beta$ to $\Sigma'$, since, we need to omit either at least $2$ out of $3$ elements of $\Sigma$ to be able to represent the structure of $M$, therefore, either $\alpha$ or $\beta$ can not be surjective.

To prove that $M$ and $M'$ are monomorphic we only have to provide a state machine homomorphism that is injective. For this we map $a$ onto $a'$ and $q\textsubscript{4}$ onto $r\textsubscript{1}$. $\alpha$ is injective because there only exists a single element in the relation, therefore, there can only exists a one-to-one mapping. For $\beta$ this is the same case as only the tuple $(q\textsubscript{4}, r\textsubscript{1})$ exists in $\beta$. 
=======
Lets assume that $\alpha$ maps to $Q'$ and $\beta$ to $\Sigma'$.
This means that we have to map the states of $Q$ to the states of $Q'$.
The state $r\textsubscript{1}$ is isolated whilst the states $r\textsubscript{2}$ and $r\textsubscript{3}$ form a cyclic state machine with stern with stern length and cycle length of $1$.
Therefore, we would need to find a similar structure in $Q$ to map the states to $Q'$.
If we look for these structures in $M$ we can find an isolated state $q\textsubscript{4}$ when we disregard $c$.
However, we can only find a cyclic state machine with stern if we either disregard $a,c$ or $b,c$ since the combination of $b$ and $c$ allows us to always move between the states of $q\textsubscript{i}$ $| i \in {3}$.
Thus, we can not find a mapping of $\beta$ to $\Sigma'$, since, we need to omit either at least $2$ out of $3$ elements of $\Sigma$ to be able to represent the structure of $M$, therefore, either $\alpha$ or $\beta$ can not be surjective.

To prove that $M$ and $M'$ are monomorphic we only have to provide a state machine homomorphism that is injective. For this we map $a$ onto $a'$ and $q\textsubscript{4}$ onto $r\textsubscript{1}$. $\alpha$ is injective because there only exists a single element in the relation, therefore, there can only exists a one-to-one mapping.
For $\beta$ this is the same case as only the tuple $(q\textsubscript{4}, r\textsubscript{1})$ exists in $\beta$. 
>>>>>>> 55c3a640b193821d34f66700df679210644d300c
Now we need to show that
\begin{equation}
\forall q \in Q \land \forall a \in \Sigma : \alpha(q\delta\textsubscript{a}) \subseteq (\alpha(q))\delta'\textsubscript{$\beta(a)$}
\end{equation}

<<<<<<< HEAD
Since, there only exist one element with one state we can apply the transition function for the element $a$ on the state $q\textsubscript{4}$ and then apply $\alpha$. $\delta\textsubscript{a}$ on $q\textsubscript{4}$ results in $q\textsubscript{4}$, and if we then apply $\alpha$ this results in $r\textsubscript{1}$. If we first apply $\alpha$ on $q\textsubscript{4}$ this results in $r\textsubscript{1}$. And if we then apply $\delta'\textsubscript{$\beta(a)$}$ on $r\textsubscript{1}$ it stays in $r\textsubscript{1}$ which proves that it is a state machine morphism. 
=======
Since, there only exist one element with one state we can apply the transition function for the element $a$ on the state $q\textsubscript{4}$ and then apply $\alpha$.
$\delta\textsubscript{a}$ on $q\textsubscript{4}$ results in $q\textsubscript{4}$, and if we then apply $\alpha$ this results in $r\textsubscript{1}$.
If we first apply $\alpha$ on $q\textsubscript{4}$ this results in $r\textsubscript{1}$.
And if we then apply $\delta'\textsubscript{$\beta(a)$}$ on $r\textsubscript{1}$ it stays in $r\textsubscript{1}$ which proves that it is a state machine morphism. 
>>>>>>> 55c3a640b193821d34f66700df679210644d300c

This concludes that $M$ and $M'$ are monomorphic.
\qed

\section{}
<<<<<<< HEAD

For Exercise 2 we need to prove that $TS(M')$ = $\overline{\mbox{TS(M)}}$ holds. The closure of a transformation semigroup $A = (Q, S)$ is defined as 
\begin{equation}
    \overline{\mbox{A}} = (Q, \langle S \cup \overline{\mbox{Q}} \rangle
\end{equation}
.
For $TS(M')$ = $\overline{\mbox{TS(M)}}$ to hold true, $TS(M')$ has to include all possible transformations from $S$ and $Q$. Since, there exists a $\delta\textsubscript{a}$ for all $a \in \Sigma$ as well as the identity function for each state of q with $\delta'\textsubscript{a}(q) = a$ for all $a \in Q$ there can not exist any other transformations in $TS(M')$ that are not included in the closure of $TS(M) = \overline{\mbox{TS(M)}}$. Therefore, $TS(M')$ = $\overline{\mbox{TS(M)}}$ holds true.
\qed
=======
We show that $\overline{TS(M)}=TS(M')=(Q,S')$. By definition
\begin{align*}
    \overline{TS(M)}=(Q,\lsk S \cup \overline{Q} \rsk)
\end{align*}
$S'$ must contain for all $q \in Q$ the corresponding $\overline{q}$.
Notice $S'= \Sigma^+/\sim_{M'}$.
We assume that for each $q \in Q \subseteq \Sigma$ exists a congruence classes $[q] \in S'$.

Proof:
Due to the definition of $\delta'_a$ for all $q_1,q_2 \in Q \subseteq \Sigma$ with $q_1 \neq q_2$ follows $\delta'_{q_1}(q) \neq \delta'_{q_2}(q)$ for all $q \in Q$.
Therefore, for all $q \in Q$ exists $[q]$ and $[q] \in \Sigma^+/\sim_{M'}$.

Since $\lsk F(M') \rsk \simeq S(M')$ for all $q \in Q$ also $\delta'_q \in F(M')$.
Finally $\delta'_q = \overline{q}$ because $\delta'_q$ maps all states to $q$ and is therefore a constant mapping which proves that $\overline{Q} \subseteq S'$ and further $\overline{TS(M)}=(Q,\lsk S \cup \overline{Q} \rsk)=TS(M')=(Q,\lsk S' \cup \overline{Q} \rsk)$.
\qed
\section{}
\section{}
>>>>>>> 55c3a640b193821d34f66700df679210644d300c
\end{document}
