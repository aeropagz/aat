\documentclass[a4paper,12pt,numbers=noenddot]{scrreport}

\usepackage[german]{datetime2}
\usepackage[onehalfspacing]{setspace}
\usepackage[utf8]{inputenc}
\usepackage{amsmath, amsfonts, amssymb, amsthm}
\usepackage{mathrsfs}

\usepackage{scrlayer-scrpage}
\renewcommand*{\chapterpagestyle}{scrheadings}
\clearpairofpagestyles

\ohead{\normalfont \today}
\chead{\normalfont Algebraic Automata Theory}
\ihead{\normalfont \rightmark}

\ofoot{\normalfont Seite~\pagemark}
\cfoot{\normalfont CAU Kiel - Technische Fakultät}
\ifoot{\normalfont Eric Hotho, Klaas Pelzer}

\KOMAoptions{headsepline=true,footsepline=true}
\renewcommand*{\chapterheadstartvskip}{\vspace*{-.4cm}}
\renewcommand*{\chapterheadendvskip}{\vspace{.5cm}}

\setlength{\parindent}{0pt}

\setkomafont{chapter}{\LARGE}
\setkomafont{section}{\Large}
\setkomafont{subsection}{\large}
\setkomafont{subsubsection}{\normalsize}
\setkomafont{paragraph}{\normalsize}
\setkomafont{subparagraph}{\small}
\usepackage[htt]{hyphenat}
\usepackage[german]{babel}
\setcounter{chapter}{3}

\def\lsk{\left<}
\def\rsk{\right>}
\DeclareMathOperator{\mmod}{mod}

\begin{document}
\automark{section}
\automark{chapter}

\setcounter{chapter}{3}
\chapter{}
\section{}

We can find a state machine homomorphism (SMH) for $M$ and $M'$ which is neither monomorphic nor epimorphic and therefore not isomorphic.
To prove this conjecture we will first show that there can not exist a epimorphic SMH.
For a SMH $f$ to be epimorphic it has to be surjective.
This means that given $f=(\alpha, \beta)$:
\begin{equation}
    \alpha : Q \rightarrow Q' \land \beta : \Sigma \rightarrow \Sigma'
\end{equation}
the range of $\alpha$ and $\beta$ have to be equal to $Q'$ and $\Sigma'$ respectively.

Let's suppose that $\alpha$ maps to $Q'$ and $\beta$ to $\Sigma'$.
This means that we have to map the states of $Q$ to the states of $Q'$.
The state $r\textsubscript{1}$ is isolated whilst the states $r\textsubscript{2}$ and $r\textsubscript{3}$ form a cyclic state machine with stern with stern length and cycle length of $1$.
Therefore, we would need to find a similar structure in $Q$ to map the states to $Q'$.
If we look for these structures in $M$ we can find an isolated state $q\textsubscript{4}$ when we disregard $c$.
However, we can only find a cyclic state machine with stern if we either disregard $a,c$ or $b,c$ since the combination of $b$ and $c$ allows us to always move between the states of $q\textsubscript{i}$ $| i \in {3}$.
Thus, we can not find a mapping of $\beta$ to $\Sigma'$, since, we need to omit either at least $2$ out of $3$ elements of $\Sigma$ to be able to represent the structure of $M$, therefore, either $\alpha$ or $\beta$ can not be surjective.

To prove that $M$ and $M'$ are not monomorphic we have to show there is no injective SMH.
Since, $M'$ has less states as well as a smaller alphabet and a homomorphism must be total, some states of $M$ have to map to the same state in $M'$ which contradicts injectivity.

Now we show that $M$ and $M'$ are morphic by providing $(\alpha, \beta)$.
\begin{align*}
    \alpha:\ &q \mapsto r_1 \ \forall q \in Q \\
    \beta:\ &u \mapsto a'\ \forall u \in \Sigma
\end{align*}
Since, we map all states to the same state in $M'$ and all letters to the same letter in $\Sigma'$ it does not matter if we either first compute in $M$ and then transform into $M'$ or first transform into $M'$ and then compute there.
We always end up in $r_1$.
This concludes that $M$ and $M'$ are morphic but not monomorphic, epimorphic or isomorphic.
\qed

\vspace{1cm}
As stated above, $M_1$ can only be monomorphic to $M_2$ if $|Q_1| \leq |Q_2|$ and there exists a one to one mapping with the homomorphism criteria.
Same applies to $\Sigma_1$ and $\Sigma_2$.

Moreover, $M_1$ can only be epimorphic to $M_2$ if $|Q_1| \geq |Q_2|$.
Otherwise due right uniquness we can not map one state of $M_1$ to two or more states in $M_2$ and therefore can not hit the whole range of $Q_2$.
This also applies to the alphabets.

For an isomorphismus both conditions of a monomorphismus and epimorphismus must apply. Thus, we have $|Q_1| = |Q_2|$ and $|\Sigma_1| = |\Sigma_2|$.
\section{}

For Exercise 2 we need to prove that $TS(M')$ = $\overline{\mbox{TS(M)}}$ holds. The closure of a transformation semigroup $A = (Q, S)$ is defined as 
\begin{equation}
    \overline{\mbox{A}} = (Q, \langle S \cup \overline{\mbox{Q}} \rangle
\end{equation}
.
For $TS(M')$ = $\overline{\mbox{TS(M)}}$ to hold true, $TS(M')$ has to include all possible transformations from $S$ and $Q$. Since, there exists a $\delta\textsubscript{a}$ for all $a \in \Sigma$ as well as the identity function for each state of q with $\delta'\textsubscript{a}(q) = a$ for all $a \in Q$ there can not exist any other transformations in $TS(M')$ that are not included in the closure of $TS(M) = \overline{\mbox{TS(M)}}$. Therefore, $TS(M')$ = $\overline{\mbox{TS(M)}}$ holds true.
\qed

% \section{}
% We show that $\overline{TS(M)}=TS(M')=(Q,S')$. By definition
% \begin{align*}
%     \overline{TS(M)}=(Q,\lsk S \cup \overline{Q} \rsk)
% \end{align*}
% $S'$ must contain for all $q \in Q$ the corresponding $\overline{q}$.
% Notice $S'= \Sigma^+/\sim_{M'}$.
% We assume that for each $q \in Q \subseteq \Sigma$ exists a congruence classes $[q] \in S'$.
%
% Proof:
% Due to the definition of $\delta'_a$ for all $q_1,q_2 \in Q \subseteq \Sigma$ with $q_1 \neq q_2$ follows $\delta'_{q_1}(q) \neq \delta'_{q_2}(q)$ for all $q \in Q$.
% Therefore, for all $q \in Q$ exists $[q]$ and $[q] \in \Sigma^+/\sim_{M'}$.
%
% Since $\lsk F(M') \rsk \simeq S(M')$ for all $q \in Q$ also $\delta'_q \in F(M')$.
% Finally $\delta'_q = \overline{q}$ because $\delta'_q$ maps all states to $q$ and is therefore a constant mapping which proves that $\overline{Q} \subseteq S'$ and further $\overline{TS(M)}=(Q,\lsk S \cup \overline{Q} \rsk)=TS(M')=(Q,\lsk S' \cup \overline{Q} \rsk)$.
% \qed
\section{}
Given $\mathbb{Z}_2$ we obtain the transformation semigroup $(\mathbb{Z}_2,\mathbb{Z}_2)$ with right multiplication as an action (Lemma 3.28).
Let $a,b \in \mathbb{Z}_2$ and notice $\mathbb{Z}_2 = \{[0],[1]\}$.
We compute $a+b$ by using the action and therefore adding its corresponding equivalence classes $[a] + [b]$.
To obtain the equivalence classes we use the mudolo operator:
\begin{align*}
    &[a] + [b]\\
    =\ &a\mmod2 + b \mmod2 \\
    =\ &a + b \mmod2 \\
\end{align*}
By definition of the mudolo operator $a \mmod2 + b \mmod2 = a + b \mmod2$ which is equal to $\delta(a,b)$.
\qed

\section{}
To prove that every group acts on itself we need show compatibility and faithfulness for a function $f: G \times G \rightarrow G; (g,g') \mapsto g'g$ with G being a group.
Compatibility follows  due to groups being associative.
Let $g_1,g_2 \in G$.
Let $gg_1 = gg_2 (0)$ for all $g \in G$ suppose $g_1 \neq g_2$.
To fulfill the equation $g$ has to be the inverse element of $g_1$ and $g_2$.
But because $g$ is fixed it also has to be the same element.
Furthermore the inverse element exists only for one other element in the group.
Thus, $g$ can not be the inverse of $g_1$ and $g_2$ at the same time.
This contradicts $g_1 \neq g_2$.
This concludes the proof.
\end{document}
