\documentclass[a4paper,12pt,numbers=noenddot]{scrreport}

\usepackage[german]{datetime2}
\usepackage[onehalfspacing]{setspace}
\usepackage[utf8]{inputenc}
\usepackage{amsmath, amsfonts, amssymb, amsthm}
\usepackage{mathrsfs}

\usepackage{scrlayer-scrpage}
\renewcommand*{\chapterpagestyle}{scrheadings}
\clearpairofpagestyles

\ohead{\normalfont \today}
\chead{\normalfont Algebraic Automata Theory}
\ihead{\normalfont \rightmark}

\ofoot{\normalfont Seite~\pagemark}
\cfoot{\normalfont CAU Kiel - Technische Fakultät}
\ifoot{\normalfont Eric Hotho, Klaas Pelzer}

\KOMAoptions{headsepline=true,footsepline=true}
\renewcommand*{\chapterheadstartvskip}{\vspace*{-.4cm}}
\renewcommand*{\chapterheadendvskip}{\vspace{.5cm}}

\setlength{\parindent}{0pt}

\setkomafont{chapter}{\LARGE}
\setkomafont{section}{\Large}
\setkomafont{subsection}{\large}
\setkomafont{subsubsection}{\normalsize}
\setkomafont{paragraph}{\normalsize}
\setkomafont{subparagraph}{\small}
\usepackage[htt]{hyphenat}
\usepackage[german]{babel}
\setcounter{chapter}{5}

\def\lsk{\left<}
\def\rsk{\right>}
\DeclareMathOperator{\mmod}{mod}

\begin{document}
\automark{section}
\automark{chapter}

\chapter{}
\section{}
\section{}
\section{}
\section{}
We show $M \leq M'$ by using the following covering $(\eta, \xi)$:
\begin{align*}
    \eta: Q' \hookrightarrow Q; q' \mapsto &
        \begin{cases}
            q_0,      & \text{for } q' = r_0, \\
            q_1,      & \text{for } q' = r_2, \\
        \end{cases} \\
    \xi: \Sigma \rightarrow \Sigma'; u \mapsto &
        \begin{cases}
            b',      & \text{for } u = a, \\
            a',      & \text{for } u = b, \\
        \end{cases}
\end{align*}
Moreover we prove $\eta, \xi$ is well defined.
First, $\xi$ is a function because all $dom(\xi) = \Sigma$ and it is right unique.
$\eta$ is surjective as its range hits all states of $Q$.
Also it is right unique and therefore a partial function.
Finally, we show $\eta(q')\delta_w \subseteq \eta(q'\delta'_{\xi(w)})$ for all $q' \in Q'$ and $w \in \Sigma^*$.
We build an operation table to get $\Sigma^+ / \sim_M$ to get all relevant $w$.

\begin{center}
\begin{tabular}{c|cc|l}
    x & $q_{0}$ & $q_{1}$ & \\ \hline
    a & $q_1$ & $\bot$ & \\ 
    b & $q_0$ & $q_1$ & \\ \hline
    aa & $\bot$ & $\bot$ & \\
    ab & $q_1$ & $\bot$ & same as a\\
    bb & $q_0$ & $q_1$ & same as b\\ 
    ba & $q_1$ & $\bot$ & same as a \\ \hline
    aaa & $\bot$ & $\bot$ & same as aa \\
    aab & $\bot$ & $\bot$ & same as aa \\
    aba & $\bot$ & $\bot$ & same as aa \\
    abb & $q_1$ & $\bot$ & same as a \\ 
    bba & $q_1$ & $\bot$ & same as a \\ 
    baa & $q_1$ & $\bot$ & same as aa \\ 
    bab & $q_1$ & $\bot$ & same as a \\ 
\end{tabular}
\end{center}
We get $\Sigma^+ / \sim_M = \{[a], [b], [aa]\}$.
Since $\eta(q') = \emptyset | q' \notin \mathcal{D}(\eta)$ we only observe for $r_0, r_2$ as states from $M'$.
We check for all relevant $q' \in Q'$ and $w in \Sigma$:
\begin{align*}
\eta(r_0)\delta_a = q_0\delta_a = q_1 \subseteq & \eta(r_0\delta_{\xi(a)}') = \eta(r_0\delta_{b'}') = \eta(r_2) = q_1 \\
\eta(r_0)\delta_b = q_0\delta_b = q_0 \subseteq & \eta(r_0\delta_{\xi(b)}') = \eta(r_0\delta_{a'}') = \eta(r_0) = q_0 \\
\eta(r_0)\delta_{aa} = q_0\delta_{aa} = \emptyset \subseteq & \eta(r_0\delta_{\xi(aa)}') = \eta(r_2\delta_{b'b'}') = \eta(\emptyset) = \emptyset \\
\eta(r_2)\delta_a = q_1\delta_a = \emptyset \subseteq & \eta(r_2\delta_{\xi(a)}') = \eta(r_2\delta_{b'}') = \eta(r_3) = \emptyset \\
\eta(r_2)\delta_b = q_1\delta_b = q_1 \subseteq & \eta(r_2\delta_{\xi(b)}') = \eta(r_2\delta_{a'}') = \eta(r_2) = q_1 \\
\eta(r_2)\delta_{aa} = q_1\delta_{aa} = \emptyset \subseteq & \eta(r_2\delta_{\xi(aa)}')  = \eta(r_2\delta_{b'b'}') = \eta(\emptyset) = \emptyset\\
\end{align*}
For each combination $\eta(q')\delta_w \subseteq \eta(q'\delta'_{\xi(w)})$ holds which concludes the proof.
\qed
\end{document}

