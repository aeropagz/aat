\documentclass[a4paper,12pt,numbers=noenddot]{scrreport}

\usepackage[german]{datetime2}
\usepackage[onehalfspacing]{setspace}
\usepackage[utf8]{inputenc}
\usepackage{amsmath, amsfonts, amssymb, amsthm}
\usepackage{mathrsfs}

\usepackage{scrlayer-scrpage}
\renewcommand*{\chapterpagestyle}{scrheadings}
\clearpairofpagestyles

\ohead{\normalfont \today}
\chead{\normalfont Algebraic Automata Theory}
\ihead{\normalfont \rightmark}

\ofoot{\normalfont Seite~\pagemark}
\cfoot{\normalfont CAU Kiel - Technische Fakultät}
\ifoot{\normalfont Eric Hotho, Klaas Pelzer}

\KOMAoptions{headsepline=true,footsepline=true}
\renewcommand*{\chapterheadstartvskip}{\vspace*{-.4cm}}
\renewcommand*{\chapterheadendvskip}{\vspace{.5cm}}

\setlength{\parindent}{0pt}

\setkomafont{chapter}{\LARGE}
\setkomafont{section}{\Large}
\setkomafont{subsection}{\large}
\setkomafont{subsubsection}{\normalsize}
\setkomafont{paragraph}{\normalsize}
\setkomafont{subparagraph}{\small}
\usepackage[htt]{hyphenat}
\usepackage[german]{babel}
\setcounter{chapter}{5}

\def\lsk{\left<}
\def\rsk{\right>}
\DeclareMathOperator{\mmod}{mod}

\begin{document}
\automark{section}
\automark{chapter}

\chapter{}
\section{}
\begin{enumerate}
    \item An epimorphism $(\alpha, \beta): M_1 \rightarrow M_2$ with complete machines implies $M_2 \leq M_1$: \\
We choose $\alpha : Q_1 \rightarrow Q_2$ as our $\eta: Q_1 \rightarrow Q_2$.
We define $\xi: \Sigma_2 \rightarrow \Sigma_1$ as a function which picks an arbitrary element of $\xi_{a_2} = \{a_1 | \beta(a_1) = a_2, a_1 \in \Sigma_1\}$ for an arbitraty $a_2 \in \Sigma_2$ (1).
There exists at least one $a_1$ for each $a_2$ because $\beta$ is surjective (2).
Since $(\alpha,\beta)$ is a state machine homomorphismus and the maschines are complete following holds for all $a_1 \in \Sigma_1$ and all $q_1 \in Q_1$ (Definition 3.37):
$$
\alpha(q_1\delta^1_{a_1}) = (\alpha(q_1))\delta^2_{\beta(a_1)}
$$ 
Now we show: $\eta(q_1)\delta^2_{a_2} \subseteq \eta(q_1\delta^1_{\xi(a_2)})$ for all $q_1 \in Q_1$ and $a_2 \in \Sigma_2$.

\begin{align*}
    \eta(q_1)\delta^2_{a_2} = & \alpha(q_1)\delta^2_{a_2} \\
                 = & \alpha(q_1)\delta^2_{\beta(a_1)} \label{eq1}\tag{2} \\
                 = & \alpha(q_1\delta^1_{a_1}) \\
                 = & \alpha(q_1\delta^1_{\xi(a_2)})\label{eq2}\tag{1} \\
                 = & \eta(q_1\delta^1_{\xi(a_2)})
\end{align*}
    \item An monomorphismus $(\alpha, \beta): M_1 \rightarrow M_2$ with complete machines implies $M_1 \leq M_2$: \\
        We choose $\beta: \Sigma_1 \rightarrow \Sigma_2$ as our $\xi: \Sigma_1 \rightarrow \Sigma_2$.
        We define $\eta: Q_2 \rightarrow Q_1$ as the inverse function from $\alpha: Q_1 \rightarrow Q_2$ which is well defined (Lemma 2.20).
        For each $q_2 \in Q_2$ with $\alpha^{-1}(q_2) \neq \emptyset$ there exists exactly one $q_1 = \alpha^{-1}(q_2)$ because $\alpha$ is injective.
        We show: $\eta(q_2)\delta^1_{a_1} \subseteq \eta(q_2\delta^2_{\xi(a_1)})$ for all $q_2 \in Q_2$ and $a_1 \in \Sigma_1$.
    \begin{align*}
        \eta(q_2)\delta^1_{a_1} = & \alpha^{-1}(q_2)\delta^1_{a_1} \\
                     = & \eta(q_2\delta^2_{\xi(a_1)})
    \end{align*}
\end{enumerate}

\section{}
For the covering relation to be a partial order it has to be \textit{reflexive, antisymmetric} and \textit{transitive}.\\
\textbf{Proof:}\\
Lets begin by showing that $\le$ is reflexive, for this let $M = (Q, \Sigma, \delta)$. We need to show that $M \le M$ is true. For $M$ to cover itself, there needs to exist a partial function $\eta: Q \rightarrow Q$ which is \begin{enumerate}
    \item surjective
    \item $\eta(q')\delta_w = \eta(q'\delta'_w)$ for all $w \in \Sigma^*$ and all $q' \in \mathcal{D}(\eta)$
    \item $\eta(q') = \emptyset$ if $q' \notin \mathcal{D}(\eta)$
    \item $q\delta_w = \emptyset$ if $ \delta(q, w)$ is undefined
\end{enumerate}
We use $id_Q$ for $\eta$, since, we do not need to change anything for $M$ to cover it self. $id_Q$ is surjective, because the identity function maps every element in $Q$ onto the same element in $Q$, therefore, it hits all elements in $Q$. Since, $\delta_w = \delta'_w$ the second primitive holds. The third holds for the identity function, since, it can only map an element to itself and if $q'$ is not in its domain (here $Q$ itself) it can not map it onto anything thus $id_Q(q') = \emptyset$ if $q' \notin \mathcal{D}(\eta)$. And the fourth holds because $M$ is complete.\\
Next we show that $\le$ is \textit{antisymmetric}:\\ Let $M_1 = (Q, \Sigma, \delta), M_2=(Q', \Sigma, ,\delta'$ be two state machines with $M_1 \le M_2 \land M_2 \le M_1$. We need to show that for this $M_1 = M_2$. Since, $M_1 \le M_2 \land M_2 \le M_1$ we know there exists two partial functions $\eta : Q' \rightarrow Q \land \eta' : Q \rightarrow Q'$ which are surjective. Furthermore, we know that 
$\eta(q')\delta_w = \eta(q'\delta'_w) $ for all $ w\in \Sigma^* $ and all $ q' \in \mathcal{D}(\eta)$ and $\eta(q)\delta'_w = \eta(q\delta_w)$ for all $w \in \Sigma^*$ and all $q \in \mathcal{D}(\eta')$ 

\section{}
\section{}
We show $M \leq M'$ by using the following covering $(\eta, \xi)$:
\begin{align*}
    \eta: Q' \hookrightarrow Q; q' \mapsto &
        \begin{cases}
            q_0,      & \text{for } q' = r_0, \\
            q_1,      & \text{for } q' = r_2, \\
        \end{cases} \\
    \xi: \Sigma \rightarrow \Sigma'; u \mapsto &
        \begin{cases}
            b',      & \text{for } u = a, \\
            a',      & \text{for } u = b, \\
        \end{cases}
\end{align*}
Moreover we prove $\eta, \xi$ is well defined.
First, $\xi$ is a function because all $dom(\xi) = \Sigma$ and it is right unique.
$\eta$ is surjective as its range hits all states of $Q$.
Also it is right unique and therefore a partial function.
Finally, we show $\eta(q')\delta_w \subseteq \eta(q'\delta'_{\xi(w)})$ for all $q' \in Q'$ and $w \in \Sigma^*$.
We build an operation table to get $\Sigma^+ / \sim_M$ to get all relevant $w$.

\begin{center}
\begin{tabular}{c|cc|l}
    x & $q_{0}$ & $q_{1}$ & \\ \hline
    a & $q_1$ & $\bot$ & \\ 
    b & $q_0$ & $q_1$ & \\ \hline
    aa & $\bot$ & $\bot$ & \\
    ab & $q_1$ & $\bot$ & same as a\\
    bb & $q_0$ & $q_1$ & same as b\\ 
    ba & $q_1$ & $\bot$ & same as a \\ \hline
    aaa & $\bot$ & $\bot$ & same as aa \\
    aab & $\bot$ & $\bot$ & same as aa \\
    aba & $\bot$ & $\bot$ & same as aa \\
    abb & $q_1$ & $\bot$ & same as a \\ 
    bba & $q_1$ & $\bot$ & same as a \\ 
    baa & $q_1$ & $\bot$ & same as aa \\ 
    bab & $q_1$ & $\bot$ & same as a \\ 
\end{tabular}
\end{center}
We get $\Sigma^+ / \sim_M = \{[a], [b], [aa]\}$.
Since $\eta(q') = \emptyset | q' \notin \mathcal{D}(\eta)$ we only observe for $r_0, r_2$ as states from $M'$.
We check for all relevant $q' \in Q'$ and $w in \Sigma$:
\begin{align*}
\eta(r_0)\delta_a = q_0\delta_a = q_1 \subseteq & \eta(r_0\delta_{\xi(a)}') = \eta(r_0\delta_{b'}') = \eta(r_2) = q_1 \\
\eta(r_0)\delta_b = q_0\delta_b = q_0 \subseteq & \eta(r_0\delta_{\xi(b)}') = \eta(r_0\delta_{a'}') = \eta(r_0) = q_0 \\
\eta(r_0)\delta_{aa} = q_0\delta_{aa} = \emptyset \subseteq & \eta(r_0\delta_{\xi(aa)}') = \eta(r_2\delta_{b'b'}') = \eta(\emptyset) = \emptyset \\
\eta(r_2)\delta_a = q_1\delta_a = \emptyset \subseteq & \eta(r_2\delta_{\xi(a)}') = \eta(r_2\delta_{b'}') = \eta(r_3) = \emptyset \\
\eta(r_2)\delta_b = q_1\delta_b = q_1 \subseteq & \eta(r_2\delta_{\xi(b)}') = \eta(r_2\delta_{a'}') = \eta(r_2) = q_1 \\
\eta(r_2)\delta_{aa} = q_1\delta_{aa} = \emptyset \subseteq & \eta(r_2\delta_{\xi(aa)}')  = \eta(r_2\delta_{b'b'}') = \eta(\emptyset) = \emptyset\\
\end{align*}
For each combination $\eta(q')\delta_w \subseteq \eta(q'\delta'_{\xi(w)})$ holds which concludes the proof.
\qed
\end{document}

