\documentclass[a4paper,12pt,numbers=noenddot]{scrreport}

\usepackage[german]{datetime2}
\usepackage[onehalfspacing]{setspace}
\usepackage[utf8]{inputenc}
\usepackage{amsmath, amsfonts, amssymb, amsthm}
\usepackage{mathrsfs}

\usepackage{scrlayer-scrpage}
\renewcommand*{\chapterpagestyle}{scrheadings}
\clearpairofpagestyles

\ohead{\normalfont \today}
\chead{\normalfont Algebraic Automata Theory}
\ihead{\normalfont \rightmark}

\ofoot{\normalfont Seite~\pagemark}
\cfoot{\normalfont CAU Kiel - Technische Fakultät}
\ifoot{\normalfont Eric Hotho, Klaas Pelzer}

\KOMAoptions{headsepline=true,footsepline=true}
\renewcommand*{\chapterheadstartvskip}{\vspace*{-.4cm}}
\renewcommand*{\chapterheadendvskip}{\vspace{.5cm}}

\setlength{\parindent}{0pt}

\setkomafont{chapter}{\LARGE}
\setkomafont{section}{\Large}
\setkomafont{subsection}{\large}
\setkomafont{subsubsection}{\normalsize}
\setkomafont{paragraph}{\normalsize}
\setkomafont{subparagraph}{\small}
\usepackage[htt]{hyphenat}
\usepackage[german]{babel}
\setcounter{chapter}{6}

\def\lsk{\left<}
\def\rsk{\right>}
\DeclareMathOperator{\mmod}{mod}

\begin{document}
\automark{section}
\automark{chapter}

\chapter{}
\section{}
We prove $M \land N \leq M \times N$. Notice for $M \land N$ if $\Sigma_m \neq \Sigma_n$ then $M \land N = \emptyset$ which is always a subset of any set.
Therefore, we assume from now on $\Sigma_m = \Sigma_n$.
Also checkout these definitions:
\begin{align*}
    (q_m, q_n) (\delta_m \land \delta_n)_a = & (q_m \delta^m_a, q_n \delta^n_a)  \label{eq1}\tag{1}\\
    (q_m, q_n) (\delta_m \times \delta_n)_{(a_m, a_n)} = & (q_m \delta^m_{a_m}, q_n \delta^n_{a_n})  \label{eq2}\tag{2}\\
\end{align*}
We choose for the covering $(\eta,\xi)$ $id_{(q_m, q_n)}$ as $\eta$ and define $\xi: \Sigma_n \rightarrow \Sigma_m \times \Sigma_n; a_m \mapsto (a_m, a_m)$.
which is well-defined due the assumption $\Sigma_n = \Sigma_m$.

Now we show $\eta((q_m, q_n)) (\delta^m \land \delta^n)_{a_m} \subseteq \eta((q_m,q_n) (\delta^m \times \delta^n)_{\xi(a_m)})$:
\begin{align*}
    \eta((q_m, q_n)) (\delta^m \land \delta^n)_{a_m} = & (q_m, q_n) (\delta^m \land \delta^n)_{a_m} \tag{def. $\eta$, 1}\\
                               = & (q_m \delta^m_{a_m}, q_n \delta^n_{a_m}) \tag{2}\\
                               = & (q_m, q_n) (\delta_m \times \delta_n)_{(a_m, a_m)} \tag{def. $\xi$}\\
                               = & (q_m, q_n) (\delta_m \times \delta_n)_{\xi(a_m)} \tag{def. $\eta$}\\
                               = & \eta((q_m,q_n) (\delta^m \times \delta^n)_{\xi(a_m)}) \\
\end{align*}
\qed
\section{}
\section{}
We prove $P \leq N \omega R$ under the assumption $M \leq N$ and $P \leq M \omega R$.
Let $\eta_1: Q_n \rightarrow Q_m$ and $\eta_2: Q_m \times Q_r \rightarrow Q_p$ be the $\eta$-functions of $M \leq N$ and $P \leq M \omega R$.
Futhermore let $\xi_1: \Sigma_p \rightarrow \Sigma_r$ be the $\xi$-function of $P \leq M \omega R$.

We define $\eta : Q_n \times Q_r \rightarrow Q_p; (q_n, q_r) \mapsto \eta_2(\eta_1(q_n), q_r)$.
Moreover we choose $\xi_1$ as $\xi$.

Finally, we show $\eta((q_n,q_r))\delta^p_{a_p} \subseteq \eta((q_n, q_r) \delta^\omega_{\xi(a_p)})$.
\begin{align*}
    \eta((q_n,q_r))\delta^p_{a_p} = & \eta_2((\eta_1 (q_n),q_r))\delta^p_{a_p} \tag{def. $\eta, \eta_1$} \\
                      = & \eta_2(q_m,q_r))\delta^p_{a_p} \tag{def. $\eta_2$} \\
                      = & q_p\delta^p_{a_p} \tag{$P \leq M \omega R$} \\
                      \subseteq & \eta_2((q_m, q_r) \delta^\omega_{\xi(a_p)}) \tag{def. $\xi$}\\
                      = & \eta_2((q_m, q_r) \delta^\omega_{a_r}) \tag{def. $\delta^\omega$}\\
                      = & \eta_2((q_m \delta^m_{\omega_{a_r}(q_r)}, q_r \delta^r_{a_r})) \tag{def. $\omega$}\\
                      = & \eta_2((q_m \delta^m_{a_m}, q_r \delta^r_{a_r})) \tag{$M \leq N$}\\
                      = & \eta_2((\eta_1(q_n \delta^n_{a_n}), q_r \delta^r_{a_r})) \tag{def. $\eta$}\\
                      = & \eta((q_n \delta^n_{a_n}, q_r \delta^r_{a_r})) \tag{def. $\omega$}\\
                      = & \eta((q_n \delta^n_{\omega_{a_r}(q_r)}, q_r \delta^r_{a_r})) \tag{def. $\delta^\omega$}\\
                      = & \eta((q_n, q_r) \delta^\omega_{a_r})) \tag{def. $\xi$}\\
                      = & \eta((q_n, q_r) \delta^\omega_{\xi(a_p)}))
\end{align*}
This gets boring...
\qed
\section{}
\end{document}

