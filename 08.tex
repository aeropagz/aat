\documentclass[a4paper,12pt,numbers=noenddot]{scrreport}

\usepackage[german]{datetime2}
\usepackage[onehalfspacing]{setspace}
\usepackage[utf8]{inputenc}
\usepackage{amsmath, amsfonts, amssymb, amsthm}
\usepackage{mathrsfs}

\usepackage{scrlayer-scrpage}
\renewcommand*{\chapterpagestyle}{scrheadings}
\clearpairofpagestyles

\ohead{\normalfont \today}
\chead{\normalfont Algebraic Automata Theory}
\ihead{\normalfont \rightmark}

\ofoot{\normalfont Seite~\pagemark}
\cfoot{\normalfont CAU Kiel - Technische Fakultät}
\ifoot{\normalfont Eric Hotho, Klaas Pelzer}

\KOMAoptions{headsepline=true,footsepline=true}
\renewcommand*{\chapterheadstartvskip}{\vspace*{-.4cm}}
\renewcommand*{\chapterheadendvskip}{\vspace{.5cm}}

\setlength{\parindent}{0pt}

\setkomafont{chapter}{\LARGE}
\setkomafont{section}{\Large}
\setkomafont{subsection}{\large}
\setkomafont{subsubsection}{\normalsize}
\setkomafont{paragraph}{\normalsize}
\setkomafont{subparagraph}{\small}
\usepackage[htt]{hyphenat}
\usepackage[german]{babel}
\setcounter{chapter}{7}

\def\lsk{\left<}
\def\rsk{\right>}
\DeclareMathOperator{\mmod}{mod}

\begin{document}
\automark{section}
\automark{chapter}

\chapter{}
\section{}
In this task we have a state machine $M = (Q, \Sigma, \delta)$. Given that, we want to prove that $G = \{x \in S(M)|Qx = Q\}$ is a group, assuming $G$ is non-empty. 
We start with the closure property:
Let $x,y \in G$, we need to show that $x \circ y \in G$, since, $x, y \in G$ implies $Qx= Q \land Qy = Q$ this means that we can apply $x$ or $y$ to any state in $Q$ and remain in $Q$. If we now compose the two functions $Q(x \circ y) = Q$ holds because composing two transformations that preserve Q, will also preserve Q, thus the closure holds. 
Next we show that G is a semigroup containing a neutral element:\\
We equip $G$ with a the closed binary operation function composition $\circ$ : $(G, \circ)$, for this to be a semigroup we need to show that it is associative: Let $a,b,c \in G$, we need to show that
\begin{equation}
    (a \circ b) \circ c = a \circ (b \circ c)
\end{equation}
holds. Since, we are concatenating input words for $S(M)$ this holds true. Next we need to show that there exists a neutral element:
\begin{equation}
    e \circ x = x \circ e = x
\end{equation}
In the context of state machines, e is the identity transformation, since $Qe = Q, e \in G$ it satisfies the identity property.
Next we need to show, that there exists an inverse element for each element in G. Since $Qx = Q$, x must be a bijective transformation on Q, therefore, there has to exist an inverse for every element because bijective functions always have inverses.
This concludes the proof that $G$ is a group. 

\section{}
Let $M = (Q, \Sigma, \delta)$, $P \subseteq Q$, $M|_P=(P, \Sigma, \delta|_P)$ and $\delta|_P: P \times \Sigma \rightarrow \Sigma$ is undefinded for all $q \in Q\setminus P$.
We define $S(M|_P) = \Sigma^+ / \sim_{M|_P}$ with $u \sim_{M|_P} v$ iff $\delta_u|_P = \delta_v|_P$.
To obtain $S(M|_P)$ you can build an operation table with $\delta|_P$ and all $p \in P$ for all $a \in \Sigma^+$.

Finally, we construct $TS(M|_P) = (P, S(M|_P))$.
\qed
\section{}
\section{}
We prove for a given function $g \in S^T$, $(S^T \times T, \wr)$ is a semigroup.
Notice $\wr$ is a closed binary operation by definition as it is a function and is closed:
\begin{align*}
    \wr: (S^T \times T) \times (S^T \times T) \rightarrow (S^T \times T)
\end{align*}
So we just have to show $\wr$ is associative:
\begin{align*}
    ((f,t) \wr (g, t')) \wr (h, t'') &= (fg_{t}, tt') \wr (h, t'') \\
                                     &= (fg_{t'}h_{tt'}, tt't''') \\
    (f,t) \wr ((g,t') \wr (h, t'')) &= (f,t) \wr (gh_{t'}, t't'') \\
                                    &= (fg_{t}h_{tt'}, tt't'')
\end{align*}
Notice $gh_{t'} = x \mapsto g(x)h(xt')$ and $(gh_{t'})_t = x \mapsto g(xt)h(xt't) = g_th_{tt'}$.
\qed
\end{document}

