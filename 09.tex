\documentclass[a4paper,12pt,numbers=noenddot]{scrreport}

\usepackage[german]{datetime2}
\usepackage[onehalfspacing]{setspace}
\usepackage[utf8]{inputenc}
\usepackage{amsmath, amsfonts, amssymb, amsthm}
\usepackage{mathrsfs}

\usepackage{scrlayer-scrpage}
\renewcommand*{\chapterpagestyle}{scrheadings}
\clearpairofpagestyles

\ohead{\normalfont \today}
\chead{\normalfont Algebraic Automata Theory}
\ihead{\normalfont \rightmark}

\ofoot{\normalfont Seite~\pagemark}
\cfoot{\normalfont CAU Kiel - Technische Fakultät}
\ifoot{\normalfont Eric Hotho, Klaas Pelzer}

\KOMAoptions{headsepline=true,footsepline=true}
\renewcommand*{\chapterheadstartvskip}{\vspace*{-.4cm}}
\renewcommand*{\chapterheadendvskip}{\vspace{.5cm}}

\setlength{\parindent}{0pt}

\setkomafont{chapter}{\LARGE}
\setkomafont{section}{\Large}
\setkomafont{subsection}{\large}
\setkomafont{subsubsection}{\normalsize}
\setkomafont{paragraph}{\normalsize}
\setkomafont{subparagraph}{\small}
\usepackage[htt]{hyphenat}
\usepackage[german]{babel}
\setcounter{chapter}{8}

\def\lsk{\left<}
\def\rsk{\right>}
\DeclareMathOperator{\mmod}{mod}

\begin{document}
\automark{section}
\automark{chapter}

\chapter{}
\section{}
We prove that if $M \leq N_1 \omega_1 N_2 ... N_{n-1}\omega_{n-1} N_n$ then $TS(M) \leq TS(N_1) \wr ... \wr TS(N_n)$.
By 4.10 
$$TS(M) \leq TS(N_1 \omega_1 N_2 ... N_{n-1}\omega_{n-1} N_n)$$ holds true.
By 6.22 the cascade product is covered by the wreath product which results in 
$$TS(N_1 \omega_1 N_2 ... N_{n-1}\omega_{n-1} N_n) \leq TS(N_1 \wr ... \wr N_n)$$.
This concludes the proof.
\section{}
We show $A \leq A/\pi \times A/\tau$.
Notice, with $S', S'' \subseteq S$:
\begin{align*}
    A &= (Q, S) \\
    A/\pi \times A/\tau &= (\pi, S') \times (\tau, S'') \rightarrow (\pi \times \tau, S' \times S'');\\
    &(q', s') \times (q'', s'') \mapsto (q's', q''s'')
\end{align*}

We define $\eta: (\pi \times \tau) \rightarrow Q; (g, h) \mapsto g \cap h$ which is surjective partial and results into one singleton of $Q$ or is the emptyset due to the orthogonal property.
$$\pi \cap \tau = id_Q$$
Finally, we show $\eta((g,h))s \subseteq \eta((g,h)(s',s''))$ with $g \in \pi, h \in \tau, s' \in S_{\sim_\pi}, s'' \in S_{\sim_\tau}$.
\begin{align*}
    \eta((g,h))s &= qs \\
                 &= q' \tag{1} \\
                 & \subseteq \eta((g', h')) \tag{2} \\
                 &= \eta((gs', hs'')) \\
                 &= \eta((g,h)(s',s''))
\end{align*}
$(1)$ By definition of $\eta$ there exists for each $q \in Q$ one block of each partition that maps to $q$ or is empty which is still valid.

$(2)$ By definition of admissible partitions for any $g' \in H$ there exists an $s' \in S/\sim$ and a partition $g \in H$ with $g' = gs'$.
\section{}
Let $M = (Q, \Sigma, \delta)$ be a reset machine with at least two states. The claim is that then for all $q_1, q_2 \in Q$, the partition $\pi = \{\{q_1, q_2\}, Q\backslash\{q_1, q_2\}\}$ is admissible and orthogonal.
Using Lemma $7.13$ we can skip the proof that $\pi$ is an admissible partition, and we only need to show that it is orthogonal. To show this we need to find another admissible partition $\tau$ of $M$ s.t. $\pi \cap \tau = id_Q$.\\
Lets look at the characteristics that $\tau$ needs to have in order for the intersection to yield only $id_Q$. For this, there can not exist $q_1$ and $q_2$ can not be together in one block of the partition, since its intersection would yield $q_1$ and $q_2$ as a result. Therefore, we take the partition WLOG. $\tau = \{\{q_1, q_3\}, \{q_2\}, \{q_i\}_{i \in (|Q| - 3)\backslash \{1,2,3\}}\}$, if we intersect $\tau$ with $\pi$, we will only get singleton blocks as a result because:
\begin{align*}
    \{q_1, q_2\}                                    &\cap \{q_1, q_3\} = \{q_1\} \\
    \{q_1, q_2\}                                    &\cap \{q_2\} = \{q_2\} \\
    \{Q\backslash\{q_1, q_2\}\}                     &\cap \{q_1, q_3\} = \{q_3\}\\ 
    \{q_1, q_2\}                                    &\cap \{q_i\}_{i \in (|Q| - 3)\backslash \{1,2,3\}} = \{\}\\
    \{q_i\}_{i \in (|Q| - 3)\backslash \{1,2,3\}}   &\cap \{Q\backslash\{q_1, q_2\}\} = \{q_i\}_{i \in (|Q| - 3)\backslash \{1,2,3\}}\\
    ~\\
    \bigcup \pi &\cap \tau = Q  
\end{align*}
And since $M$ is a reset machine, all partitions of it are admissible and therefore, also $\tau$. This concludes the proof.
\qed
\section{}
We use the partition $\pi = \{\{q_0, q_2,q_4\}, \{q_1, q_3\}\}$ and $\tau = \{\{q_0, q_1\}, \{q_2, q_3\}, \{q_4\}\}$. First we show that both $\pi$ and $\tau$ are admissible partitions and after that we show that they are orthogonal.\\
\begin{align*}
    \{q_0, q_2,q_4\}\delta_a    &= \{q_0, q_2,q_4\}\\
    \{q_0, q_2,q_4\}\delta_b    &= \{q_1, q_3\}\\
    \{q_1, q_3\}\delta_a        &= \{q_1, q_3\}\\
    \{q_1, q_3\}\delta_b        &= \{q_1, q_3\}\\
    ~\\
    \{q_0, q_1\}\delta_a        &= \{q_2, q_3\}\\
    \{q_0, q_1\}\delta_b        &= \{q_0, q_1\}\\
    \{q_2, q_3\}\delta_a        &= \{q_4\}\\
    \{q_2, q_3\}\delta_b        &= \{q_0, q_1\}\\
    \{q_4\}\delta_a             &= \{q_2, q_3\}\\
    \{q_4\}\delta_b             &= \{q_2, q_3\}\\
\end{align*}
Next we show that $\pi \cap \tau = id_Q$:\\
\begin{align*}
    \{q_0, q_2,q_4\}    &\cap \{q_0, q_1\}  &= \{q_0\}\\
    \{q_0, q_2,q_4\}    &\cap \{q_2, q_3\}  &= \{q_2\}\\
    \{q_0, q_2,q_4\}    &\cap \{q_4\}       &= \{q_4\}\\
    \{q_1, q_3\}        &\cap \{q_0, q_1\}  &= \{q_1\}\\
    \{q_1, q_3\}        &\cap \{q_2, q_3\}  &= \{q_3\}\\
    \{q_1, q_3\}        &\cap \{q_4\}       &= \{\}\\
\end{align*}
This concludes the proof. 
\qed
\end{document}

