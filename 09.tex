\documentclass[a4paper,12pt,numbers=noenddot]{scrreport}

\usepackage[german]{datetime2}
\usepackage[onehalfspacing]{setspace}
\usepackage[utf8]{inputenc}
\usepackage{amsmath, amsfonts, amssymb, amsthm}
\usepackage{mathrsfs}

\usepackage{scrlayer-scrpage}
\renewcommand*{\chapterpagestyle}{scrheadings}
\clearpairofpagestyles

\ohead{\normalfont \today}
\chead{\normalfont Algebraic Automata Theory}
\ihead{\normalfont \rightmark}

\ofoot{\normalfont Seite~\pagemark}
\cfoot{\normalfont CAU Kiel - Technische Fakultät}
\ifoot{\normalfont Eric Hotho, Klaas Pelzer}

\KOMAoptions{headsepline=true,footsepline=true}
\renewcommand*{\chapterheadstartvskip}{\vspace*{-.4cm}}
\renewcommand*{\chapterheadendvskip}{\vspace{.5cm}}

\setlength{\parindent}{0pt}

\setkomafont{chapter}{\LARGE}
\setkomafont{section}{\Large}
\setkomafont{subsection}{\large}
\setkomafont{subsubsection}{\normalsize}
\setkomafont{paragraph}{\normalsize}
\setkomafont{subparagraph}{\small}
\usepackage[htt]{hyphenat}
\usepackage[german]{babel}
\setcounter{chapter}{8}

\def\lsk{\left<}
\def\rsk{\right>}
\DeclareMathOperator{\mmod}{mod}

\begin{document}
\automark{section}
\automark{chapter}

\chapter{}
\section{}
We prove that if $M \leq N_1 \omega_1 N_2 ... N_{n-1}\omega_{n-1} N_n$ then $TS(M) \leq TS(N_1) \wr ... \wr TS(N_n)$.
By 4.10 
$$TS(M) \leq TS(N_1 \omega_1 N_2 ... N_{n-1}\omega_{n-1} N_n)$$ holds true.
By 6.22 the cascade product is covered by the wreath product which results in 
$$TS(N_1 \omega_1 N_2 ... N_{n-1}\omega_{n-1} N_n) \leq TS(N_1 \wr ... \wr N_n)$$.
This concludes the proof.
\section{}
\section{}
\section{}
We can find admissable partitions for example:
$$\pi = \{\{q_0\}, \{q_1, q_2, q_3, q_4\}\}, \tau = \{\{q_1, q_3\}, \{q_0, q_2, q_4\}\}$$
However we never find an orthogonal partition with $\pi \cap \tau = id_Q$ because for an input of multiple $b \in \Sigma$ all transistions will eventually end in $q_1 \in Q$ where is no escape, thus $\pi \cap \tau = \emptyset$.
Therefore the only partition with the $id_Q$ function is the trivial partition of whole $Q$ which is a contradiction.
\qed
\end{document}

