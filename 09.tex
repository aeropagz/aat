\documentclass[a4paper,12pt,numbers=noenddot]{scrreport}

\usepackage[german]{datetime2}
\usepackage[onehalfspacing]{setspace}
\usepackage[utf8]{inputenc}
\usepackage{amsmath, amsfonts, amssymb, amsthm}
\usepackage{mathrsfs}

\usepackage{scrlayer-scrpage}
\renewcommand*{\chapterpagestyle}{scrheadings}
\clearpairofpagestyles

\ohead{\normalfont \today}
\chead{\normalfont Algebraic Automata Theory}
\ihead{\normalfont \rightmark}

\ofoot{\normalfont Seite~\pagemark}
\cfoot{\normalfont CAU Kiel - Technische Fakultät}
\ifoot{\normalfont Eric Hotho, Klaas Pelzer}

\KOMAoptions{headsepline=true,footsepline=true}
\renewcommand*{\chapterheadstartvskip}{\vspace*{-.4cm}}
\renewcommand*{\chapterheadendvskip}{\vspace{.5cm}}

\setlength{\parindent}{0pt}

\setkomafont{chapter}{\LARGE}
\setkomafont{section}{\Large}
\setkomafont{subsection}{\large}
\setkomafont{subsubsection}{\normalsize}
\setkomafont{paragraph}{\normalsize}
\setkomafont{subparagraph}{\small}
\usepackage[htt]{hyphenat}
\usepackage[german]{babel}
\setcounter{chapter}{8}

\def\lsk{\left<}
\def\rsk{\right>}
\DeclareMathOperator{\mmod}{mod}

\begin{document}
\automark{section}
\automark{chapter}

\chapter{}
\section{}
We prove that if $M \leq N_1 \omega_1 N_2 ... N_{n-1}\omega_{n-1} N_n$ then $TS(M) \leq TS(N_1) \wr ... \wr TS(N_n)$.
By 4.10 
$$TS(M) \leq TS(N_1 \omega_1 N_2 ... N_{n-1}\omega_{n-1} N_n)$$ holds true.
By 6.22 the cascade product is covered by the wreath product which results in 
$$TS(N_1 \omega_1 N_2 ... N_{n-1}\omega_{n-1} N_n) \leq TS(N_1 \wr ... \wr N_n)$$.
This concludes the proof.
\section{}
\section{}
Let $M = (Q, \Sigma, \delta)$ be a reset machine with at least two states. The claim is that then for all $q_1, q_2 \in Q$, the partition $\pi = \{\{q_1, q_2\}, Q\backslash\{q_1, q_2\}\}$ is admissible and orthogonal.
Using Lemma $7.13$ we can skip the proof that $\pi$ is an admissible partition, and we only need to show that it is orthogonal. To show this we need to find another admissible partition $\tau$ of $M$ s.t. $\pi \cap \tau = id_Q$.\\
Lets look at the characteristics that $\tau$ needs to have in order for the intersection to yield only $id_Q$. For this, there can not exist $q_1$ and $q_2$ can not be together in one block of the partition, since its intersection would yield $q_1$ and $q_2$ as a result. Therefore, we take the partition WLOG. $\tau = \{\{q_1, q_3\}, \{q_2\}, \{q_i\}_{i \in (|Q| - 3)\backslash \{1,2,3\}}\}$, if we intersect $\tau$ with $\pi$, we will only get singleton blocks as a result because:
\begin{align*}
    \{q_1, q_2\}                                    &\cap \{q_1, q_3\} = \{q_1\} \\
    \{q_1, q_2\}                                    &\cap \{q_2\} = \{q_2\} \\
    \{Q\backslash\{q_1, q_2\}\}                     &\cap \{q_1, q_3\} = \{q_3\}\\ 
    \{q_1, q_2\}                                    &\cap \{q_i\}_{i \in (|Q| - 3)\backslash \{1,2,3\}} = \{\}\\
    \{q_i\}_{i \in (|Q| - 3)\backslash \{1,2,3\}}   &\cap \{Q\backslash\{q_1, q_2\}\} = \{q_i\}_{i \in (|Q| - 3)\backslash \{1,2,3\}}\\
    ~\\
    \bigcup \pi &\cap \tau = Q  
\end{align*}
And since $M$ is a reset machine, all partitions of it are admissible and therefore, also $\tau$. This concludes the proof.
\qed
\section{}
We can find admissible partitions for example:
$$\pi = \{\{q_0\}, \{q_1, q_2, q_3, q_4\}\}, \tau = \{\{q_1, q_3\}, \{q_0, q_2, q_4\}\}$$
However we never find an orthogonal partition with $\pi \cap \tau = id_Q$ because $q_2, q_4 \in Q$ are tightly coupled such that in each partition $q_2, q_4$ are in the same block.
Therefore we never find $\pi, \tau$ with $\pi \cap \tau = id_Q$ because ${q_2, q_4}$ is always a subset of the intersection.
\qed
\end{document}

